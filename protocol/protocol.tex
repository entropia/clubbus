\documentclass[a4paper]{scrartcl}

\usepackage[T1]{fontenc}
\usepackage[utf8]{inputenc}
\usepackage{hyperref}
\usepackage{lmodern}
\usepackage{microtype}
\usepackage{multirow}

\title{\EDEN{} Protocol}
\author{florolf}

\newcommand{\EDEN}{{\sc Eden}}

\begin{document}

\maketitle
\tableofcontents

\section{The network}
\EDEN{} uses CAN as its basis. CAN guarantees fault-tolerant in order delivery.

\section{Address format}
\EDEN{} frames use CAN extended IDs (29 bits) which are divided as
follows:

\begin{figure}[h]
  \centering
  \begin{tabular}{|l|l|l|l|l|}
    \hline
    \texttt{0} & \texttt{111111111111} &
    \texttt{111111111111} & \texttt{1111} \\
    type: direct & destination (12bit) & source (12bit) & port (4bit) \\
    \hline
    \texttt{1} & \texttt{0} & \multicolumn{2}{|l|}{\texttt{111111111111111111111111111}} \\
    type: event & globally assigned & \multicolumn{2}{|l|}{event ID (27 bit)} \\
    \hline
    \texttt{1} & \texttt{1} & \texttt{111111111111} &
    \texttt{111111111111111} \\
    type: event & locally assigned & by node & event ID \\
    \hline
  \end{tabular}
  \caption{\EDEN{} address structure}
  \label{fig:eden-eid}
\end{figure}

Every participating \EDEN{} node has an unique 12 bit ID, which is used
in direct communication. In addition, communication to a node is
multiplexed using a 4 bit port number.

\EDEN{} nodes can also broadcast data in response to specific
events. Event IDs are centrally assigned and no event ID must be used
by more than one node for CAN arbitration to work.

\subsection{Special use}
The special case of $\texttt{src} \equiv \texttt{dst} \equiv \texttt{node
  ID}$ is used in communication between the bootloader and bootflashd
to conserve space in the bootloader.

\section{Common protocol}
The following common protocol is established:

\begin{figure}[h]
  \centering
  \begin{tabular}{lc}
    In: & \texttt{command arguments} \\
    Out: & \texttt{status-code response}
  \end{tabular}
\end{figure}

All integers shall be represented in big-endian byte order.

\section{Management}
Port number 0 is reserved for management applications.

\subsection{Ping}
\begin{tabular}{ll}
  In: & \texttt{00} $\texttt{data}_{0-7}$ \\
  Out (success): & \texttt{00 data} \\
  Out (failure): & \texttt{status}
\end{tabular}
\vspace{5pt}

Returns any data back to the sender.

\subsection{SW-Version}
\begin{tabular}{ll}
  In: & \texttt{01} \\
  Out (success): & \texttt{00} ${\texttt{git-revision}|}_7$ \\
  Out (failure): & \texttt{status}
\end{tabular}
\vspace{5pt}

Returns the git revision number of the nodes firmware truncated to the
first 8 bytes.

\subsection{Reset}
\begin{tabular}{ll}
  In: & \texttt{02} \\
  Out (success): & --- \\
  Out (failure): & \texttt{status}
\end{tabular}
\vspace{5pt}

Reboot the node.

\subsection{Read EEPROM}
\begin{tabular}{ll}
  In: & \texttt{03} $\texttt{offset}_{2}$ $\texttt{length}_{1}$ \\
  Out (success): & \texttt{00 data} \\
  Out (failure): & \texttt{status}
\end{tabular}
\vspace{5pt}

Read data from the EEPROM, if available. \texttt{length} must be $\leq
7$

\subsection{Write EEPROM}
\begin{tabular}{ll}
  In: & \texttt{04} $\texttt{offset}_{2}$ $\texttt{data}_{1-6}$ \\
  Out: & \texttt{status}
\end{tabular}
\vspace{5pt}

Write data to the EEPROM.

\subsection{Get MCP inputs}
\begin{tabular}{ll}
  In: & \texttt{05} \\
  Out: & \texttt{status} $\texttt{values}_1$
\end{tabular}
\vspace{5pt}

Get the values of the MPC's inputs. (0-1 are wired to jumpers, 2 is floating).

\subsection{Set MCP outputs}
\begin{tabular}{ll}
  In: & \texttt{06} $\texttt{mask}_{1}$ $\texttt{values}_{1}$ \\
  Out: & \texttt{status}
\end{tabular}
\vspace{5pt}

If the nth bit of \texttt{mask} is \texttt{1}, set the corresponding output according
to the nth bit of \texttt{values}.

\subsection{Get current MCP outputs}
\begin{tabular}{ll}
  In: & \texttt{07} \\
  Out: & \texttt{status} $\texttt{values}_1$
\end{tabular}
\vspace{5pt}

Report the current state of the MCP's outputs.

\section{Logging}

Logs are sent from nodes to a configured log receiver on port 15 (\texttt{0xF}) in plaintext chunks
of up to 8 Bytes. Lines are terminated by a NUL-byte and may be up to 128 characters
long (including the NUL-byte).

\section{Status codes}

Status codes $< \texttt{0x80}$ are considered to be well-known,
status codes $\geq \texttt{0x80}$ are defined by the respective
function that was called.


\begin{figure}[h]
  \centering
  \begin{tabular}{|c|c|}
    \hline
    \texttt{0} & OK \\
    \hline
    \texttt{1} & Not implemented \\
    \hline
    \texttt{2} & Invalid argument \\
    \hline
  \end{tabular}
  
  \caption{Well-known \EDEN{} status codes}
\label{fig:eden-codes}
\end{figure}


\end{document}
